\documentclass{exam}
\usepackage[utf8]{inputenc}
\usepackage{systeme}
\usepackage{amsmath}

\title{MATH Recommender Systems Deliverables}
\date{May 13th, 2017}

\begin{document}
\maketitle
\section{Summary}
There will be two parts that need to be submitted, a report and the code used. The report should be done in either Latex or Word. The plots for the report can be generated using a tool like matplotlib in Python or they can be done manually in Excel. Using matplotlib is recommended. The code should be done using Eider or a similar tool. 

\section{Report}
\subsection{Data Preparation}
The report should cover the rational used in the Data Preparation step. Describe how a data-set should be verified. Describe any normalization that was done to the data. Describe the data structures you used to represent the data in your code. 

\subsection{Modeling the problem}
This part of the report should contain a general explanation of the model used. That includes:
\begin{enumerate}
\item What value is used for k? Why is that an appropriate choice?
\item What function is used for the objective function? Why was this function chosen? How is it minimized?
\item Describe the use of randomness to initialize U and V.
\item Explain the logic behind using a training set and a test set.
\item Plot of error function over iterations. Use iterations along the x-axis and the values of the error function along the y-axis.
\end{enumerate}

\subsection{Computing gradients}
Plot the norm of the gradient with respect to U and V over iterations. Use iterations along the x-axis and the norm along the y-axis

\subsection{Metrics}
Describe the metric used for evaluating the model on the test set.

\subsection{Summary}
Summarize the experience and what was learned out of this machine learning project including any modeling, mathematical and algorithmic insights.

\section{Code}
The code should include everything you used for the project. It should be in a form that would make it possible for other people to generate the data needed for your report by “replaying” your code. An Eider notebook is recommended (eider.corp.amazon.com).
\end{document}
